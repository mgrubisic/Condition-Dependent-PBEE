\chapter{INTRODUCTION}
\label{chap-one}

Bridges are designed based on discrete events with minimal consideration of interactions between hazards/loading, material aging (or more accurately condition) and bridge performance. The purpose of the research described is to study Time Dependent Performance Based Design that considers the effects of cumulative damage on the properties of the materials both as a function of time and current condition. Specific items of interest include corrosion, strain aging, low cycle fatigue and strength aging. In addition, since there is a high likelihood for a structure  in a high seismic region to be subjected to more than one main shock throughout its life, it is deemed important to consider the effects of multiple earthquakes. As a consequence, the effects of repairs on the structural response are also of great importance. An analytical procedure is implemented such that it considers the effect of aging on structures, more specifically this study starts by evaluating an RC bridge Column. A series of condition dependent nonlinear time history analysis are performed assuming that a series of earthquakes occurs throughout the lifetime of the structure while at the same time changing the properties of the structure as time progresses. To achieve this a library of time dependent materials are developed. At the end of each series the main variables of study are the the limit state that was reached, the controlling mode of response (flexural or shear controlled), Equivalent Viscous Damping and the accumulated deformations. The series of earthquake proposed consists of (1) equally spaced main shocks only, (2) main shock-aftershocks series and (3) main shock-aftershock-repair series. At the end of the presentation recommendations on design of new structures and assessment of existing structures will be provided.

\section{Motivation}
\lipsum[1]
\section{Scope and layout}
\lipsum[2]